\hypertarget{docModule_classCreation}{}\section{Création d'une nouvelle classe}\label{docModule_classCreation}
Pour créer un nouveau macro-\/module, il faut hériter une nouvelle classe de la classe \hyperlink{classMacroModule}{Macro\-Module}. {\itshape Remarque\-: la classe \hyperlink{classMacroModule}{Macro\-Module} est héritée de \hyperlink{classModule}{Module}, un macro-\/module est donc un module.}

Par exemple pour créer le macro-\/module {\bfseries \hyperlink{classBatteryModule}{Battery\-Module}}\-: {\ttfamily class \hyperlink{classBatteryModule}{Battery\-Module}\-:public \hyperlink{classMacroModule}{Macro\-Module}\{\};}

Il faut également au moins redéfinir deux methodes\-:
\begin{DoxyEnumerate}
\item le constructeur Macro\-Module\-::\-Macro\-Module
\item Macro\-Module\-::process (qui est héritée de Module\-::process)
\end{DoxyEnumerate}

Notez que le constructeur de votre nouveau module doit appeller le constructeur de base de la classe \hyperlink{classMacroModule}{Macro\-Module}\-:


\begin{DoxyCode}
\hyperlink{classBatteryModule_a2fb494ef5f124c38c0fdf9ccfb31918f}{BatteryModule::BatteryModule}(std::string name, Params params) : 
      \hyperlink{classMacroModule}{MacroModule}(name, params, \textcolor{stringliteral}{"BatteryModule/BatteryModule.xml"})\{
    \textcolor{comment}{/* ce qu'il faut faire pendant la création d'un module */}
\}
\end{DoxyCode}


Les arguments de ce constructeur de base sont les suivants\-:
\begin{DoxyEnumerate}
\item name \-: le nom de votre macro-\/module.
\item params \-: un objet de type {\ttfamily std\-::unordered\-\_\-map$<$std\-::string, double$>$} avec des paramètres d'un macro-\/module.
\item \char`\"{}\-Battery\-Module/\-Battery\-Module.\-xml\char`\"{} – le chemin vers le fichier {\ttfamily .xml} qui contient la configuration du module.
\end{DoxyEnumerate}\hypertarget{docModule_properties}{}\section{Propriétés d'un Module}\label{docModule_properties}
\hypertarget{docMacroModule_submodules}{}\subsection{Sous-\/modules}\label{docMacroModule_submodules}
La difference principale entre \hyperlink{classModule}{Module} et \hyperlink{classMacroModule}{Macro\-Module} est la possibilité d'ajouter des sous-\/modules dans \hyperlink{classMacroModule}{Macro\-Module} et d'établir des liens entre eux. \hypertarget{docMacroModule_addsubmodule}{}\subsubsection{Ajout d'un sous-\/module}\label{docMacroModule_addsubmodule}
L'ajout d'un module à un macro-\/module est effectué par la méthode Macro\-Module\-::dd\-Sub\-Module. Il faut le faire dans le constructeur du macro-\/module. Exemple du constructeur \hyperlink{classBatteryModule}{Battery\-Module} avec ajout des sous-\/modules \hyperlink{classBattery}{Battery} et \hyperlink{classBatteryController}{Battery\-Controller}\-: 
\begin{DoxyCode}
\hyperlink{classBatteryModule_a2fb494ef5f124c38c0fdf9ccfb31918f}{BatteryModule::BatteryModule}(std::string name, Params params) : 
      \hyperlink{classMacroModule}{MacroModule}(name, params, \textcolor{stringliteral}{"BatteryModule/BatteryModule.xml"})\{
    \textcolor{comment}{//Les modules:}
    addSubModule(\textcolor{keyword}{new} \hyperlink{classBattery}{Battery}());
    addSubModule(\textcolor{keyword}{new} \hyperlink{classBatteryController}{BatteryController}());        \}
\end{DoxyCode}
\hypertarget{docMacroModule_connectSubModules}{}\subsubsection{Etablissement des connections entre sous-\/modules}\label{docMacroModule_connectSubModules}
La création de connections entre sous-\/modules se fait par la méthode \hyperlink{classMacroModule_a6ad4d6abd8bb4b742b800f6fa3c98296}{Macro\-Module\-::add\-Connexion}. Exemple de constructeur \hyperlink{classBatteryModule}{Battery\-Module} avec établissement des connections entre sous-\/modules \hyperlink{classBatteryController}{Battery\-Controller} (en utilisant le \hyperlink{classSocket}{Socket} \char`\"{}to\-Battery\-Controller\char`\"{}) et \hyperlink{classBattery}{Battery} (en utilisant le \hyperlink{classSocket}{Socket} \char`\"{}to\-Battery\char`\"{})\-: 
\begin{DoxyCode}
\hyperlink{classBatteryModule_a2fb494ef5f124c38c0fdf9ccfb31918f}{BatteryModule::BatteryModule}(std::string name, Params params) : 
      \hyperlink{classMacroModule}{MacroModule}(name, params, \textcolor{stringliteral}{"BatteryModule/BatteryModule.xml"})\{
     addConnexion(\textcolor{keyword}{new} \hyperlink{classConnexion}{Connexion}(getModuleByName(\textcolor{stringliteral}{"Battery"})->getSocketByName(\textcolor{stringliteral}{"toBatteryController"})
      , getModuleByName(\textcolor{stringliteral}{"BatteryController"})->getSocketByName(\textcolor{stringliteral}{"toBattery"})));
\}
\end{DoxyCode}
 